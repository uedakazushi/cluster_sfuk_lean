% In this file you should put the actual content of the blueprint.
% It will be used both by the web and the print version.
% It should *not* include the \begin{document}
%
% If you want to split the blueprint content into several files then
% the current file can be a simple sequence of \input. Otherwise It
% can start with a \section or \chapter for instance.

\chapter{Introduction}

シンプレクティックコホモロジーは
ある種のシンプレクティック多様体に対して定義される
Floerコホモロジーの一種であり、
シンプレクティックトポロジーにおける基本的な不変量であるだけでなく、
ミラー対称性においても重要な役割を果たす。

シンプレクティックコホモロジーの持つ重要な性質として、
深谷圏のHochschildコホモロジーと同型であることが挙げられる。
ホモロジー的ミラー対称性によって
あるシンプレクティック多様体の深谷圏が
適当なスタックの安定導来圏と同型になる時、
そのシンプレクティック多様体のシンプレクティックコホモロジーは
安定導来圏のHochschildコホモロジーとして計算される。

超曲面特異点の
安定導来圏のHochschildコホモロジーは
Jacobi環の言葉で記述される事が知られている
(例えば\cite[Theorem 3.1]{MR4442683}とその参考文献を見よ)。
これを例えばFermat型の超曲面特異点の場合に具体的に書き下すと、
組み合わせ論的な公式を得ることができる。
そのような具体的な記述は、
一般次元の単純特異点
\begin{align} \label{eq:simple}
\begin{split}
 A_\ell &\colon x_1^{\ell+1} + x_2^2 + \cdots + x_{n+1}^2 = 0, \quad \ell = 1, 2, \ldots \\
%  \label{eq:Aell} \\
 D_\ell &\colon x_1^{\ell-1} + x_1 x_2^2 + x_3^2+ \cdots + x_{n+1}^2 = 0, \quad \ell = 4, 5, \ldots \\
%  \label{eq:Dell} \\
 E_6 &\colon x_1^4 + x_2^3 + x_3^2 + \cdots + x_{n+1}^2 = 0, \\
%  \label{eq:E6} \\
 E_7 &\colon x_1^3 + x_1 x_2^3 + x_3^2 + \cdots + x_{n+1}^2 = 0, \\
%  \label{eq:E7} \\
 E_8 &\colon x_1^5 + x_2^3 + x_3^2 + \cdots + x_{n+1}^2 = 0.
%  \label{eq:E8}
\end{split}
\end{align}
の場合に\cite{MR4371540}、
いくつかの3次元cDV特異点の場合に\cite{MR4648096}、
そして可逆多項式で定義される3次元cDV特異点の場合に\cite{2404.17301v1}
において行われた。
更に、\cite{MR4648096}では、
この計算を元に、
次のような予想が提案された:

\begin{conjecture}[{\cite[Conjecture 1.4]{MR4648096}}] \label{cj:Evans-Lekili}
\en{A compound Du Val singularity
admits a small resolution
if and only if the dimension of the symplectic cohomology of its Milnor fiber
is constant in every negative cohomological degree.
Furthermore,
if this is the case,
then this dimension is equal to the number of irreducible components
of the exceptional locus of a small resolution.}
\ja{
cDV特異点は、
そのミルナーファイバーのシンプレクティックコホモロジーの次元が
負の全ての次数で一定である場合に限り、
小さな特異点解消を持つ。
さらに、この場合、
この次元は小さな特異点解消の例外集合の既約成分の数に等しい。
}
\end{conjecture}

\en{
The following refinement of \ref{cj:Evans-Lekili}
was also proposed by the authors of \cite{MR4648096}
based on unpublished calculations:
}
\ja{
また、\Cref{cj:Evans-Lekili}の次のような精密化が、
\cite{MR4648096}の著者らによって、
未公表の計算に基づいて提案された:
}
\begin{conjecture} \label{cj:refined_Evans-Lekili}
\en{
Let $Y \to X$ be a small $\bQ$-factorialization
of a compound Du Val singularity
$P \in X$.
Let further $\ell$ be the number of irreducible components
of the exceptional locus,
$\Uv$ be the Milnor fiber of $P \in X$,
and
$\Uv_1,\ldots,\Uv_r$ be the Milnor fibers
of the resulting $\bQ$-factorial singularities
$Q_1,\ldots, Q_r \in Y$.
Then one has
\begin{align}
\dim \SH^i \left( \Uv \right)
= \sum_{j=1}^r \dim \SH^i \left( \Uv_j \right) + \ell
\end{align}
for any $i < 0$.}
\ja{
$Y \to X$をcDV特異点$P \in X$の小さな$\bQ$分解化とする。
さらに、$\ell$を例外集合の既約成分の数、
$\Uv$を$P \in X$のミルナーファイバー、
$\Uv_1,\ldots,\Uv_r$を
$\bQ$分解化の結果として得られた$\bQ$分解的特異点
$Q_1,\ldots, Q_r \in Y$のミルナーファイバーとする。
この時、任意の$i < 0$に対して、
\begin{align}
\dim \SH^i \left( \Uv \right)
= \sum_{j=1}^r \dim \SH^i \left( \Uv_j \right) + \ell
\end{align}
が成り立つ。
}
\end{conjecture}

\cite{2406.15915}では、
2以上の整数$e$と$f$に対して
\begin{align} \label{eq:cDV_ef}
x^2 + y^2 + z^{e} + w^{f} = 0
\end{align}
で定義されるcDV特異点に対して
\Cref{cj:refined_Evans-Lekili}が成り立つ事を示した。
このcDV特異点のMilnorファイバーを
$\Uv_{e,f}$と置くと、
そのシンプレクティックコホモロジーは
ホモロジー的ミラー対称性によって
ある(群作用付きの)超曲面の
安定導来圏のHochschildコホモロジーと同型であり、
後者は組合せ論的な記述を持つ。
この組合せ論的な記述をもとに、
自然数$i$に対して
$h_{e,f}^i$を以下のように定義する:

\begin{definition}
\label{df:setI}
\lean{setI}
\leanok
$
\rI_{e,f}^i
\coloneqq
\{
n \in \bN
\mid
n \not \equiv - 1 \mod e
\ 
\text{かつ}
\ 
n \not \equiv - 1 \mod f
\ 
\text{かつ}
\ 
\lfloor n / e \rfloor + \lfloor n / f \rfloor = i
\}
$
\end{definition}

\begin{definition}
\label{df:setII}
\lean{setII}
\leanok
$
\rII_{e,f}^i
\coloneqq
\{
n \in \bN
\mid
n \equiv - 1 \mod e
\ 
\text{かつ}
\ 
n \equiv - 1 \mod f
\ 
\text{かつ}
\ 
\lfloor n / e \rfloor + \lfloor n / f \rfloor + 1 = i
\}
$
\end{definition}

\begin{definition}
\label{df:setIII}
\lean{setIII}
\leanok
$
\rIII_{e,f}^i
\coloneqq
\{
\xi \in \bC
\mid
\xi \ne 1
\ 
\text{かつ}
\ 
\xi^e = \xi^f = 1
\}
$
\end{definition}

\begin{definition}
\label{df:h}
\lean{h}
\leanok
\uses{lm:setI_finite,lm:setII_finite,lm:setIII_finite}
$
h_{e,f}^i
\coloneqq
\left| \rI_{e,f}^i \right|
+ \left| \rII_{e,f}^i \times \rIII_{e,f}^i \right|
$.
\end{definition}

この時、次が成り立つ:

\begin{theorem} \label{th:a combinatorial formula for dim SH}
任意の2以上の整数$e,f$と自然数$i$に対して
\begin{align}
h_{e,f}^i
= \dim \SH^{-2i} \left( \Uv_{e,f} \right)
= \dim \SH^{-2i+1} \left( \Uv_{e,f} \right)
\end{align}
が成り立つ。
\end{theorem}

\Cref{th:a combinatorial formula for dim SH}を踏まえると、
\Cref{cj:refined_Evans-Lekili}は
次の\Cref{pr:main}に帰着される:

\begin{proposition}[{\cite[Proposition 5.1]{2406.15915}}]
\label{pr:main}
\en{
For integers $e$ and $f$ greater than two
and a non-positive integer $i$,
one has
\begin{align}
h_{e \ell, f \ell}^i = \ell h_{e,f}^i + \ell - 1.
\end{align}
}
\ja{任意の2以上の整数$e$と$f$および正でない整数$i$に対して、
\begin{align}
h_{e \ell, f \ell}^i = \ell h_{e,f}^i + \ell - 1
\end{align}
が成り立つ。}
\end{proposition}

\cite[Theorem C]{2404.17301v1}では
$h_{e,f}^i$に対する次のような公式が与えられた:

\begin{theorem}[{\cite[Theorem C]{2404.17301v2}}]
\label{th:APZ Theorem C v1}
任意の自然数$i$に対して、
$e (1+i)$を$e+f$で割った余りを
$0 \le q < e+f$と置くと、
\begin{align}
h_{e,f}^{i} =
\begin{cases}
\gcd(e,f)-1 & q = 0 \ \text{または}\ q=e+f-1, \\
q - 1 & 1 \le q \le \min \{e, f\}, \\
\min \{e, f\} - 1 & \min \{e, f\} < q < e+f-1, \end{cases}
\end{align}
が成り立つ。
\end{theorem}

この公式が正しければ、
\Cref{pr:main}が
系として従う。
しかし、
\href{https://github.com/uedakazushi/cluster_sfuk_lean/blob/master/dim_SH.ipynb}{SageMathを用いた計算}
は、
この公式が正しくないことを示唆している。
このため、我々は\cite{2406.15915}で
この公式に依存しない証明を与えた。
なお、
修正された公式が\cite[Theorem C]{2404.17301v2}において与えられている:

\begin{theorem}[{\cite[Theorem C]{2404.17301v2}}]
\label{th:APZ Theorem C v2}
任意の自然数$i$に対して、
$\min \{ e, f \} (1+i)$を$e+f$で割った余りを
$0 \le q < e+f$と置くと、
\begin{align}
h_{e,f}^{i} =
\begin{cases}
\gcd(e,f)-1 & q = 0, \\
q - 1 & 1 \le q \le \min \{e, f\}, \\
\min \{e, f\} - 1 & \min \{e, f\} < q \le \max \{e, f\}, \\
e + f - q - 1 & \max \{e, f\} < q \le e + f - 1,
\end{cases}
\end{align}
が成り立つ。
\end{theorem}

\Cref{th:APZ Theorem C v1}は極めて初等的な主張なので、
もし\Cref{th:APZ Theorem C v1}を我々が計算機で検証しなければ、
証明を詳細に検討せずに信じて、
\Cref{pr:main}に対する誤った証明を与えていたかもしれない。
また、計算機で検証する際に、
彼らの主張ではなくそれを検証するために書いた我々のコードが
間違っていた可能性もあったし、
彼らの主張が正しくなかったと分かった今でも、
我々のコードが正しいという保証はない。
このプロジェクトの目標は、
\cref{pr:main}を形式化することである。

\chapter{Preparation}

\begin{definition}
\label{df:nat_interval}
\lean{nat_interval}
\leanok
自然数$a$, $b$に対して、
\textsf{nat\_interval a b}
で
$
[a, b] \coloneqq \{n \in \bN \mid a \le n \le b\}
$
を表す。
\end{definition}

\begin{lemma}
\label{lm:nat_interval_card}
\lean{nat_interval_card}
\uses{df:nat_interval}
\leanok
$
|[a,b]| = b - a + 1
$.
\end{lemma}

\begin{lemma}
\label{lm:finite_of_bounded_of_Nat}
\lean{finite_of_bounded_of_Nat}
\leanok
\en{
A bounded set of natural numbers is finite.
}
\ja{
自然数からなる有界な集合は有限である。
}
\end{lemma}

自然数$n,m$に対して、
$n / m \in \bN$は$n$を$m$で割った商を表し、
$n \% m \in \bN$は$n$を$m$で割った余りを表す。

\begin{lemma}
\label{lm:nat_mod_pnat_le}
\en{
If $n \in \bN$ and $m \in \bN^+$,
then $n \% m \le m$.
}
\ja{
もし$n \in \bN$かつ$m \in \bN^+$ならば、
$n \% m \le m$である。
}
\end{lemma}

\begin{remark}
\en{
Note that one has $n \% m = n$ when $m = 0$,
and that the operator $\%$ in \texttt{PNat}
takes values in \texttt{PNat}
(so that
$n \% (-m) = n \% m$
for all
$n \in \bZ$
and
$m \in \bN^+$).
}
\ja{
$m=0$の時、$n \% m = n$であることと、
\texttt{PNat}における$\%$は
\texttt{PNat}に値を取る
(従って、任意の$n \in \bZ$と$m \in \bN^+$に対して
$n \% (-m) = n \% m$
である
)
ことに注意せよ。
}
\end{remark}

% \chapter{Main definitions}

\Cref{df:h}が意味をなすためには、
次の補題が必要である。

\begin{lemma}
\label{lm:setI_finite}
\lean{setI_finite}
\uses{df:setI,lm:finite_of_bounded_of_Nat,lm:nat_mod_pnat_le}
\leanok
\en{The set $\rI_{e,f}^i$ is finite.}
\ja{$\rI_{e,f}^i$は有限集合である。}
\end{lemma}

\begin{lemma}
\label{lm:setII_finite}
% \lean{setII_finite}
$\rII_{e,f}^i$は有限集合である。
\end{lemma}

\begin{lemma}
\label{lm:setIII_finite}
% \lean{setII_finite}
$\rIII_{e,f}^i$は有限集合である。
\end{lemma}
\chapter{Main result}

\en{
Let $e$, $f$, and $\ell$ be positive integers
and $i$ be a nonnegative integer.
Assume that $e$ and $f$ are coprime.
}
\ja{
$e$, $f$, $\ell$を正の整数、$i$を非負整数とする。
$e$と$f$が互いに素であると仮定する。
}

\begin{lemma} \label{lm:I_is_interval}
\en{
There exists $n_{\max}$ and $n_{\min}$ such that
$\rI_{e,f}^i = \{n \in \bN \mid n_{\min} \le n \le n_{\max}\}$.
}
\ja{
$n_{\max}$と$n_{\min}$が存在して、
$\rI_{e,f}^i = \{n \in \bN \mid n_{\min} \le n \le n_{\max}\}$
が成り立つ。
}
\end{lemma}

\chapter{Bibliography}

\def\cprime{$'$} \def\cprime{$'$}
\providecommand{\bysame}{\leavevmode\hbox to3em{\hrulefill}\thinspace}
\providecommand{\MR}{\relax\ifhmode\unskip\space\fi MR }
% \MRhref is called by the amsart/book/proc definition of \MR.
\providecommand{\MRhref}[2]{%
  \href{http://www.ams.org/mathscinet-getitem?mr=#1}{#2}
}
\providecommand{\href}[2]{#2}
\newcommand{\arXiv}[1]{\href{https://arxiv.org/abs/#1}{arXiv:#1}}

\begin{thebibliography}{APZv1}
\bibitem[APZv1]{2404.17301v1}
Nikolas Adaloglou, Federica Pasquotto, and Aline Zanardini,
\emph{Symplectic cohomology of {$cA_n$} singularities},
\arXiv{2404.17301v1}.
\bibitem[APZv2]{2404.17301v2}
Nikolas Adaloglou, Federica Pasquotto, and Aline Zanardini,
\emph{Symplectic cohomology of {$cA_n$} singularities},
\arXiv{2404.17301v2}.
\bibitem[EL23]{MR4648096}
Jonathan~David Evans and Yanki Lekili,
\emph{Symplectic cohomology of compound {D}u {V}al singularities},
Ann. H. Lebesgue \textbf{6} (2023), 727--765. \arXiv{2104.11713}
\bibitem[LU21]{MR4371540}
Yank{\i} Lekili and Kazushi Ueda, \emph{Homological mirror symmetry for {M}ilnor fibers of simple singularities}, Algebr. Geom. \textbf{8} (2021), no.~5, 562--586. \arXiv{2004.07374}
\bibitem[LU22]{MR4442683}
Yank\i Lekili and Kazushi Ueda, \emph{Homological mirror symmetry for {M}ilnor fibers via moduli of {$A_\infty$}-structures}, J. Topol. \textbf{15} (2022), no.~3, 1058--1106. \arXiv{1806.04345}
\bibitem[LU]{2406.15915}
Yank{\i} Lekili, Kazushi Ueda,
\emph{Homological mirror symmetry for Rabinowitz Fukaya categories
of Milnor fibers of Brieskorn-Pham singularities},
\arXiv{2406.15915}.
\bibitem[W]{W}
Freek Wiedijk,
\emph{The ``de {Bruijn} factor''},
last modified on 2012-03-01,
available at \url{https://www.cs.ru.nl/~freek/factor/}.
\end{thebibliography}
