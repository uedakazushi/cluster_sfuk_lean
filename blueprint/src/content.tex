% In this file you should put the actual content of the blueprint.
% It will be used both by the web and the print version.
% It should *not* include the \begin{document}
%
% If you want to split the blueprint content into several files then
% the current file can be a simple sequence of \input. Otherwise It
% can start with a \section or \chapter for instance.

% \section{Introduction}

% For integers $n$ and $e$,
% we write $n \% e$ for the remainder of the division of $n$ by $e$,
% and $n \div e$ for the integer quotient of $n$ by $e$.

\chapter{Preparation}

\begin{lemma}
\label{lem:finite_of_bounded_of_Nat}
\lean{finite_of_bounded_of_Nat}
\leanok
\en{
A bounded set of natural numbers is finite.
}
\ja{
自然数からなる有界な集合は有限である。
}
\end{lemma}

\begin{lemma}
\label{lem:nat_mod_pnat_le}
\en{
If $n \in \bN$ and $m \in \bN^+$,
then $n \% m \le m$.
}
\ja{
もし$n \in \bN$かつ$m \in \bN^+$ならば、
$n \% m \le m$である。
}
\end{lemma}

\begin{remark}
\en{
Note that one has $n \% m = n$ when $m = 0$,
and that the operator $\%$ in \texttt{PNat}
takes values in \texttt{PNat}
(so that
$n \% (-m) = n \% m$
for all
$n \in \bZ$
and
$m \in \bN^+$).
}
\ja{
$m=0$の時、$n \% m = n$であることと、
\texttt{PNat}における$\%$は
\texttt{PNat}に値を取る
(従って、任意の$n \in \bZ$と$m \in \bN^+$に対して
$n \% (-m) = n \% m$
である
)
ことに注意せよ。
}
\end{remark}

\chapter{Main definitions}

Let $e$ and $f$ be positive integers
and $i$ be a nonnegative integer.

\begin{definition}
\label{def:setI}
\lean{setI}
\leanok
$
\rI_{e,f}^i
\coloneqq
\{
n \in \bN
\mid
(n \not \equiv - 1 \mod e)
\wedge
(n \not \equiv - 1 \mod f)
\wedge
(\lfloor n / e \rfloor + \lfloor n / f \rfloor = i)
\}
$
\end{definition}

% \begin{definition}
% \label{def:setII}
% \lean{setII}
% $
% \rII_{e,f}^i
% \coloneqq
% \{
% n \in \bN
% \mid
% (n \equiv - 1 \mod e)
% \wedge
% (n \equiv - 1 \mod f)
% \wedge
% (\lfloor n / e \rfloor + \lfloor n / f \rfloor + 1 = i)
% \}
% $
% \end{definition}

\begin{lemma}
\label{lem:setI_finite}
\lean{setI_finite}
\uses{def:setI,lem:finite_of_bounded_of_Nat,lem:nat_mod_pnat_le}
\leanok
The set $\rI_{e,f}^i$ is finite.
\end{lemma}

% \begin{lemma}
% \label{lem:setII_finite}
% \lean{setII_finite}
% The set $\rII_{e,f}^i$ is finite.
% \end{lemma}

\Cref{def:h} is a variation
of the definition of $h_{e,f}^i$ in \cite{2406.15915},
which coincides with that of \cite{2406.15915}
if $i \not \equiv -1 \mod e+f$.

\begin{definition}
\label{def:h}
\lean{h}
\leanok
\uses{lem:setI_finite}
$
h_{e,f}^i
\coloneqq
|\rI_{e,f}^i|
% + |\rII_{e,f}^i|
$.
\end{definition}

\chapter{Main result}

Let $e$, $f$, and $\ell$ be positive integers
and $i$ be a nonnegative integer.
Assume that $e$ and $f$ are coprime.

\begin{proposition}[{\cite[Proposition 5.1]{2406.15915}}]
\label{prop:main}
\lean{main}
\uses{def:h,
lem:I_is_interval}
$
h_{e \ell, f \ell}^i = \ell h_{e,f}^i + \ell - 1
$
\end{proposition}

\begin{lemma} \label{lem:I_is_interval}
There exists $n_{\max}$ and $n_{\min}$ such that
$\rI_{e,f}^i = \{n \in \bN \mid n_{\min} \le n \le n_{\max}\}$.
\end{lemma}

\chapter{Bibliography}

\begin{thebibliography}{LU}
\bibitem[LU]{2406.15915}
Yank{\i} Lekili, Kazushi Ueda,
\emph{Homological mirror symmetry for Rabinowitz Fukaya categories
of Milnor fibers of Brieskorn-Pham singularities},
arXiv:2406.15915.
\end{thebibliography}