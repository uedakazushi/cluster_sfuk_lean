% In this file you should put the actual content of the blueprint.
% It will be used both by the web and the print version.
% It should *not* include the \begin{document}
%
% If you want to split the blueprint content into several files then
% the current file can be a simple sequence of \input. Otherwise It
% can start with a \section or \chapter for instance.

\chapter{Introduction}

シンプレクティックコホモロジーは
ある種のシンプレクティック多様体に対して定義される
Floerコホモロジーの一種であり、
シンプレクティックトポロジーにおける基本的な不変量であるだけでなく、
ミラー対称性においても重要な役割を果たす。

シンプレクティックコホモロジーの持つ重要な性質として、
深谷圏のHochschildコホモロジーと同型であることが挙げられる。
ホモロジー的ミラー対称性によって
あるシンプレクティック多様体の深谷圏が
適当なスタックの安定導来圏と同型になる時、
そのシンプレクティック多様体のシンプレクティックコホモロジーは
安定導来圏のHochschildコホモロジーとして計算される。

超曲面特異点の
安定導来圏のHochschildコホモロジーは
Jacobi環の言葉で記述される事が知られている
(例えば\cite[Theorem 3.1]{MR4442683}とその参考文献を見よ)。
これを例えばFermat型の超曲面特異点の場合に具体的に書き下すと、
組み合わせ論的な公式を得ることができる。
そのような具体的な記述は、
一般次元の単純特異点
\begin{align} \label{eq:simple}
\begin{split}
 A_\ell &\colon x_1^{\ell+1} + x_2^2 + \cdots + x_{n+1}^2 = 0, \quad \ell = 1, 2, \ldots \\
%  \label{eq:Aell} \\
 D_\ell &\colon x_1^{\ell-1} + x_1 x_2^2 + x_3^2+ \cdots + x_{n+1}^2 = 0, \quad \ell = 4, 5, \ldots \\
%  \label{eq:Dell} \\
 E_6 &\colon x_1^4 + x_2^3 + x_3^2 + \cdots + x_{n+1}^2 = 0, \\
%  \label{eq:E6} \\
 E_7 &\colon x_1^3 + x_1 x_2^3 + x_3^2 + \cdots + x_{n+1}^2 = 0, \\
%  \label{eq:E7} \\
 E_8 &\colon x_1^5 + x_2^3 + x_3^2 + \cdots + x_{n+1}^2 = 0.
%  \label{eq:E8}
\end{split}
\end{align}
の場合に\cite{MR4371540}、
いくつかの3次元cDV特異点の場合に\cite{MR4648096}、
そして可逆多項式で定義される3次元cDV特異点の場合に\cite{2404.17301v1}
において行われた。
更に、\cite{MR4648096}では、
この計算を元に、
次のような予想が提案された:

\begin{conjecture}[{\cite[Conjecture 1.4]{MR4648096}}] \label{cj:Evans-Lekili}
\en{A compound Du Val singularity
admits a small resolution
if and only if the dimension of the symplectic cohomology of its Milnor fiber
is constant in every negative cohomological degree.
Furthermore,
if this is the case,
then this dimension is equal to the number of irreducible components
of the exceptional locus of a small resolution.}
\ja{
cDV特異点は、
そのミルナーファイバーのシンプレクティックコホモロジーの次元が
負の全ての次数で一定である場合に限り、
小さな特異点解消を持つ。
さらに、この場合、
この次元は小さな特異点解消の例外集合の既約成分の数に等しい。
}
\end{conjecture}

\en{
The following refinement of \ref{cj:Evans-Lekili}
was also proposed by the authors of \cite{MR4648096}
based on unpublished calculations:
}
\ja{
また、\Cref{cj:Evans-Lekili}の次のような精密化が、
\cite{MR4648096}の著者らによって、
未公表の計算に基づいて提案された:
}
\begin{conjecture} \label{cj:refined_Evans-Lekili}
\en{
Let $Y \to X$ be a small $\bQ$-factorialization
of a compound Du Val singularity
$P \in X$.
Let further $\ell$ be the number of irreducible components
of the exceptional locus,
$\Uv$ be the Milnor fiber of $P \in X$,
and
$\Uv_1,\ldots,\Uv_r$ be the Milnor fibers
of the resulting $\bQ$-factorial singularities
$Q_1,\ldots, Q_r \in Y$.
Then one has
\begin{align}
\dim \SH^i \left( \Uv \right)
= \sum_{j=1}^r \dim \SH^i \left( \Uv_j \right) + \ell
\end{align}
for any $i < 0$.}
\ja{
$Y \to X$をcDV特異点$P \in X$の小さな$\bQ$分解化とする。
さらに、$\ell$を例外集合の既約成分の数、
$\Uv$を$P \in X$のミルナーファイバー、
$\Uv_1,\ldots,\Uv_r$を
$\bQ$分解化の結果として得られた$\bQ$分解的特異点
$Q_1,\ldots, Q_r \in Y$のミルナーファイバーとする。
この時、任意の$i < 0$に対して、
\begin{align}
\dim \SH^i \left( \Uv \right)
= \sum_{j=1}^r \dim \SH^i \left( \Uv_j \right) + \ell
\end{align}
が成り立つ。
}
\end{conjecture}

\cite{2406.15915}では、
2以上の整数$e$と$f$に対して
\begin{align} \label{eq:cDV_ef}
x^2 + y^2 + z^{e} + w^{f} = 0
\end{align}
で定義されるcDV特異点に対して
\Cref{cj:refined_Evans-Lekili}が成り立つ事を示した。
このcDV特異点のMilnorファイバーを
$\Uv_{e,f}$と置くと、
そのシンプレクティックコホモロジーは
ホモロジー的ミラー対称性によって
ある(群作用付きの)超曲面の
安定導来圏のHochschildコホモロジーと同型であり、
後者は組合せ論的な記述を持つ。
この組合せ論的な記述をもとに、
自然数$i$に対して
$h_{e,f}^i$を以下のように定義する:

\begin{definition}
\label{df:setI}
\lean{setI}
\leanok
$
\rI_{e,f}^i
\coloneqq
\{
n \in \bN
\mid
n \not \equiv - 1 \mod e
\ 
\text{かつ}
\ 
n \not \equiv - 1 \mod f
\ 
\text{かつ}
\ 
\lfloor n / e \rfloor + \lfloor n / f \rfloor = i
\}
$
\end{definition}

\begin{definition}
\label{df:setII}
\lean{setII}
\leanok
$
\rII_{e,f}^i
\coloneqq
\{
n \in \bN
\mid
n \equiv - 1 \mod e
\ 
\text{かつ}
\ 
n \equiv - 1 \mod f
\ 
\text{かつ}
\ 
\lfloor n / e \rfloor + \lfloor n / f \rfloor + 1 = i
\}
$
\end{definition}

\begin{definition}
\label{df:setIII}
\lean{setIII}
\leanok
$
\rIII_{e,f}^i
\coloneqq
\{
\xi \in \bC
\mid
\xi \ne 1
\ 
\text{かつ}
\ 
\xi^e = \xi^f = 1
\}
$
\end{definition}

\begin{definition}
\label{df:h}
\lean{h}
\leanok
\uses{lm:setI_finite,lm:setII_finite,lm:setIII_finite}
$
h_{e,f}^i
\coloneqq
\left| \rI_{e,f}^i \right|
+ \left| \rII_{e,f}^i \times \rIII_{e,f}^i \right|
$.
\end{definition}

この時、次が成り立つ:

\begin{theorem} \label{th:a combinatorial formula for dim SH}
任意の2以上の整数$e,f$と自然数$i$に対して
\begin{align}
h_{e,f}^i
= \dim \SH^{-2i} \left( \Uv_{e,f} \right)
= \dim \SH^{-2i+1} \left( \Uv_{e,f} \right)
\end{align}
が成り立つ。
\end{theorem}

\Cref{th:a combinatorial formula for dim SH}を踏まえると、
\Cref{cj:refined_Evans-Lekili}は
次の\Cref{pr:main}に帰着される:

\begin{proposition}[{\cite[Proposition 5.1]{2406.15915}}]
\label{pr:main}
\lean{main}
\uses{df:h,lm:nat_interval_card}
\en{
For integers $e$ and $f$ greater than two
and a non-positive integer $i$,
one has
\begin{align}
h_{e \ell, f \ell}^i = \ell h_{e,f}^i + \ell - 1.
\end{align}
}
\ja{任意の2以上の互いに素な整数$e$と$f$、正の整数$\ell$
および自然数$i$に対して、
\begin{align}
h_{e \ell, f \ell}^i = \ell h_{e,f}^i + \ell - 1
\end{align}
が成り立つ。}
\end{proposition}

\cite[Theorem C]{2404.17301v1}では
$h_{e,f}^i$に対する次のような公式が与えられた:

\begin{theorem}[{\cite[Theorem C]{2404.17301v2}}]
\label{th:APZ Theorem C v1}
任意の自然数$i$に対して、
$e (1+i)$を$e+f$で割った余りを
$0 \le q < e+f$と置くと、
\begin{align}
h_{e,f}^{i} =
\begin{cases}
\gcd(e,f)-1 & q = 0 \ \text{または}\ q=e+f-1, \\
q - 1 & 1 \le q \le \min \{e, f\}, \\
\min \{e, f\} - 1 & \min \{e, f\} < q < e+f-1, \end{cases}
\end{align}
が成り立つ。
\end{theorem}

この公式が正しければ、
\Cref{pr:main}が
系として従う。
しかし、
\href{https://github.com/uedakazushi/cluster_sfuk_lean/blob/master/dim_SH.ipynb}{SageMathを用いた計算}
は、
この公式が正しくないことを示唆している。
このため、我々は\cite{2406.15915}で
この公式に依存しない証明を与えた。
なお、
我々が誤りを指摘した後に、
修正された公式が\cite[Theorem C]{2404.17301v2}において与えられている:

\begin{theorem}[{\cite[Theorem C]{2404.17301v2}}]
\label{th:APZ Theorem C v2}
任意の自然数$i$に対して、
$\min \{ e, f \} (1+i)$を$e+f$で割った余りを
$0 \le q < e+f$と置くと、
\begin{align}
h_{e,f}^{i} =
\begin{cases}
\gcd(e,f)-1 & q = 0, \\
q - 1 & 1 \le q \le \min \{e, f\}, \\
\min \{e, f\} - 1 & \min \{e, f\} < q \le \max \{e, f\}, \\
e + f - q - 1 & \max \{e, f\} < q \le e + f - 1,
\end{cases}
\end{align}
が成り立つ。
\end{theorem}

\Cref{th:APZ Theorem C v1}は極めて初等的な主張なので、
もし\Cref{th:APZ Theorem C v1}を我々が計算機で検証しなければ、
証明を詳細に検討せずに信じて、
\Cref{pr:main}に対する誤った証明を与えていたかもしれない。
また、計算機で検証する際に、
彼らの主張ではなくそれを検証するために書いた我々のコードが
間違っていた可能性もあったし、
彼らの主張が正しくなかったと分かった今でも、
我々のコードが正しいという保証はない。
このプロジェクトの目標は、
\cref{pr:main}を形式化することと、
$h_{e,f}^i$を計算するための
形式的に検証されたコードを提供することである。

\chapter{準備}

\section{自然数の区間}

\begin{definition}
\label{df:nat_interval}
\lean{nat_interval}
\leanok
自然数$a$, $b$に対して、
\textsf{nat\_interval a b}
を
\textsf{Finset.range (b + 1) {\textbackslash} Finset.range a}
と定義し、
この青写真では$[a,b]$で表す。
\end{definition}

\begin{lemma}
\label{lm:nat_interval_mem}
\lean{nat_interval_mem}
\uses{df:nat_interval}
\leanok
任意の$a,b,c \in \bN$に対して、
$
b \in [a, c]
$
であるための必要十分条件は
$
a \le b \le c
$
である。
すなわち、
$
[a, b] \coloneqq \{n \in \bN \mid a \le n \le b\}
$
である。
\end{lemma}

\begin{definition}
\label{df:IsInterval}
\lean{IsInterval}
\uses{df:nat_interval}
\leanok
$
S \subset \bN
$
が区間であるとは、
任意の
$
a, b, c \in \bN
$
に対して、
$
a, c \in S
$
かつ
$
a \le b \le c
$
であれば
$
b \in S
$
であることをいう。
\end{definition}

\begin{lemma}
\label{lm:nonempty_interval_range}
\lean{nonempty_interval_range}
\uses{df:IsInterval}
\leanok
任意の空でない有限集合
$
S \subset \bN
$
が区間であれば、
$
S = [\min S, \max S]
$
である。
\end{lemma}

\begin{lemma}
\label{lm:preimage_of_monotone_isInterval}
\lean{preimage_of_monotone_isInterval}
\uses{df:IsInterval}
\leanok
$
f \colon \bN \to \bN
$
を単調増加関数とすると、
任意の
$
i \in \bN
$
に対して、
$
f^{-1}(i) \subset \bN
$
は区間である。
\end{lemma}

\begin{lemma}
\label{lm:nat_interval_card}
\lean{nat_interval_card}
\uses{df:nat_interval}
\leanok
$
|[a,b]| = (b+1) - a
$.
\end{lemma}

\begin{remark}
自然数の減法\textsf{Nat.sub}は、
定義から
$n \le m$に対して$n-m=0$を満たす。
\end{remark}

\begin{lemma}
\label{lm:finite_of_bounded_of_Nat}
\lean{finite_of_bounded_of_Nat}
\leanok
\en{
A bounded set of natural numbers is finite.
}
\ja{
自然数からなる有界な集合は有限である。
}
\end{lemma}

\begin{lemma}
\label{lm:fib_monotone_ubd_fun_bdd}
\lean{fib_monotone_ubd_fun_bdd}
\leanok
任意の単調増加関数
$
f \colon \bN \to \bN
$
が、
任意の自然数$n$に対して
ある自然数$i$が存在して
$
n \le f(i)
$
を満たすならば、
任意の自然数$j$に対して
$
f^{-1}(j) \subset \bN
$
は有界である。
\end{lemma}

\begin{definition}
\label{df:IsMinIn}
\lean{IsMinIn}
\leanok
自然数$m$が
自然数の集合$S \subset \bN$の最小元であるとは、
$
m \in S
$
かつ
任意の$x \in S$に対して
$
m \le x
$
が成り立つことを指す。
\end{definition}

\begin{lemma}
\label{lm:min_unique}
\lean{min_unique}
\uses{df:IsMinIn}
\leanok
自然数$m, m'$がともに
$S \subset \bN$の最小元であれば、
$m = m'$である。
\end{lemma}

\section{Lipschitz関数}

\begin{definition}
\label{df:MonotoneOneLipschitz}
\lean{MonotoneOneLipschitz}
\leanok
単調増加関数
$
f \colon \bN \to \bN
$
が1-Lipschitzであるとは、
任意の自然数$n$に対して
$
f(n + 1) \le f(n) + 1
$
が成り立つことをいう。
\end{definition}

\section{自然数の商と余り}

自然数\texttt{n}, \texttt{m}に対して
\texttt{n / m}
と
\texttt{n \% m}
は
自然数の範囲で\texttt{n}を\texttt{m}で割った
商と余り
を表す。

\begin{lemma}
\label{lm:nat_div_pnat_le}
\lean{nat_div_pnat_le}
\leanok
自然数$n,q$と
正の自然数$d$に対して、
$\lfloor n / d \rfloor \le q$ならば$n \le q \cdot d + d$が成り立つ。
\end{lemma}

% \begin{remark}
% \en{
% Note that one has $n \% m = n$ when $m = 0$,
% and that the operator $\%$ in \texttt{PNat}
% takes values in \texttt{PNat}
% (so that
% $n \% (-m) = n \% m$
% for all
% $n \in \bZ$
% and
% $m \in \bN^+$).
% }
% \ja{
% $m=0$の時、$n \% m = n$であることと、
% \texttt{PNat}における$\%$は
% \texttt{PNat}に値を取る
% (従って、任意の$n \in \bZ$と$m \in \bN^+$に対して
% $n \% (-m) = n \% m$
% である
% )
% ことに注意せよ。
% }
% \end{remark}

\begin{definition}
\label{df:φ}
\lean{φ}
\leanok
$
\varphi_{e,f} \colon \bN \to \bN
$
を
$
n \mapsto \lfloor n / e \rfloor + \lfloor n / f \rfloor
$
で定義する。
\end{definition}

\begin{lemma}
\label{lm:φ_monotone}
\lean{φ_monotone}
\uses{df:φ}
\leanok
任意の自然数$e, f$に対して、
$\varphi_{e,f}$は単調増加関数である。
\end{lemma}

% \begin{lemma}
% \label{lm:φ_mul}
% \lean{φ_mul}
% \uses{df:φ}
% \leanok
% 自然数$e,f,n$と
% 正の自然数$\ell \in$に対して
% $
% \varphi_{e \ell,f \ell}(n \ell)
% =
% \varphi_{e,f}(n)
% $.
% \end{lemma}

\begin{lemma}
\label{lm:I_as_a_subset_of_preimage_φ}
\lean{I_as_a_subset_of_preimage_φ}
\uses{df:setI,df:φ}
\leanok
任意の$e,f,i \in \bN$に対して、
$
\rI_{e,f}^i
=
\left\{
n \in \varphi_{e,f}^{-1}( i )
\relmid
n \not \equiv -1 \mod e
\ 
\text{かつ}
\ 
n \not \equiv - 1 \mod f
\right\}
$
が成り立つ。
\end{lemma}

\Cref{df:h}が意味をなすためには、
次の補題が必要である。

\begin{lemma}
\label{lm:setI_finite}
\lean{setI_finite}
\uses{df:setI,lm:finite_of_bounded_of_Nat,lm:nat_div_pnat_le}
\leanok
\en{The set $\rI_{e,f}^i$ is finite.}
\ja{$\rI_{e,f}^i$は有限集合である。}
\end{lemma}

\begin{lemma}
\label{lm:setII_finite}
\lean{setII_finite}
\uses{df:setII}
$\rII_{e,f}^i$は有限集合である。
\end{lemma}

\begin{lemma}
\label{lm:setIII_finite}
\lean{setIII_finite}
\uses{df:setIII}
$\rIII_{e,f}^i$は有限集合である。
\end{lemma}

\chapter{Auxiliary function}

\begin{lemma}
\label{lm:nat_succ_div_le}
\lean{nat_succ_div_le}
\leanok
任意の自然数$n$と正の自然数$d$に対して、
$
\lfloor (n+1)/d \rfloor
\le
\lfloor (n/d) \rfloor + 1
$
が成り立つ。
\end{lemma}

\begin{proof}
$
\lfloor (n+1)/e \rfloor
\le
\lfloor (n+e)/e \rfloor
=
\lfloor n/e \rfloor + 1
$.
\end{proof}

\begin{lemma}
\label{lm:φ_n_add_one_le_φ_n_add_two}
\lean{φ_n_add_one_le_φ_n_add_two}
\uses{lm:nat_succ_div_le,df:φ}
\leanok
$\varphi_{e,f}(n+1) \le \varphi_{e,f}(n) + 2$.
\end{lemma}

\begin{lemma}
\label{lm:φinv_i_empty_implies_φinv_i_add_one_nonempty}
\lean{φinv_i_empty_implies_φinv_i_add_one_nonempty}
\uses{lm:φ_n_add_one_le_φ_n_add_two}
任意の自然数$i$に対し、
$\varphi_{e,f}^{-1}(i) = \emptyset$
ならば
$\varphi_{e,f}^{-1}(i+1) \ne \emptyset$.
\end{lemma}

\begin{proof}
ある自然数$i$に対して
$\varphi_{e,f}^{-1}(i) = \emptyset$かつ
$\varphi_{e,f}^{-1}(i+1) = \emptyset$ならば、
$\varphi_{e,f}^{-1}(n) < i$を満たす
最大の自然数$n$に対し、
$\varphi_{e,f}(n) \le i-1$かつ
$\varphi_{e,f}(n+1) \ge i+2$が成り立つが、
これは\Cref{lm:φ_n_add_one_le_φ_n_add_two}に矛盾する。
\end{proof}

\begin{lemma}
\label{lm:φ_mul}
\lean{φ_mul}
\uses{df:φ}
\leanok
$\varphi_{e \ell, f \ell}(n \ell) = \varphi_{e,f}(n)$.
\end{lemma}

\begin{lemma}
任意の$n \in \varphi_{e,f}^{-1}(i)$
に対して
$n \ell \in \varphi_{e \ell,f \ell}^{-1}(i)$.
\end{lemma}

\begin{lemma}
\label{lm:not_dvd_mod_eq}
\lean{not_dvd_mod_eq}
\leanok
自然数$e,n$が
$n \ne 0$
かつ
$e \not | n$
を満たす時、
$\lfloor n / e \rfloor = \lfloor (n-1) / e \rfloor$
\end{lemma}

\begin{lemma}
\label{lm:φ_n_minus_one_eq_φ_n}
\lean{φ_n_minus_one_eq_φ_n}
\uses{lm:not_dvd_mod_eq,df:φ}
\leanok
自然数
$
e,f,n
$
が
$
n \ne 0
$
かつ
$
e | n
$
を満たす時、
$
\varphi_{e,f}(n-1) + 1 \le \varphi_{e,f}(n)
$
\end{lemma}

\begin{lemma}
\label{lm:dvd_mod_ne}
\lean{dvd_mod_ne}
\leanok
自然数$e,n$が
$n \ne 0$
かつ
$e | n$
を満たす時、
$\lfloor (n-1) / e \rfloor + 1 = \lfloor n / e \rfloor$.
\end{lemma}

\begin{lemma}
\label{eq:φ_n_minus_one_eq_φ_n}
\lean{φ_n_minus_one_eq_φ_n}
\uses{lm:not_dvd_mod_eq,df:φ}
自然数
$
e, f, n
$
が
$
n \ne 0
$
かつ
$
e \not | n
$
かつ
$
f \not | n
$
を満たす時、
$
\varphi_{e,f}(n-1) = \varphi_{e,f}(n)
$.
\end{lemma}

\begin{lemma}
\label{lm:φ_n_minus_one_ne_φ_n_e}
\lean{φ_n_minus_one_ne_φ_n_e}
\uses{lm:not_dvd_mod_eq,df:φ}
自然数
$
e, f, n
$
が
$
n \ne 0
$
かつ
$
e | n
$
を満たす時、
$
\varphi_{e,f}(n-1) = \varphi_{e,f}(n)
$.
\end{lemma}

\begin{lemma}
\label{lm:φ_n_minus_one_ne_φ_n_f}
\lean{φ_n_minus_one_ne_φ_n_f}
\uses{lm:not_dvd_mod_eq,df:φ}
自然数
$
e, f, n
$
が
$
n \ne 0
$
かつ
$
f | n
$
を満たす時、
$
\varphi_{e,f}(n-1) = \varphi_{e,f}(n)
$.
\end{lemma}

\begin{lemma}
\label{eq:φ_n_minus_one_ne_φ_n}
\lean{φ_n_minus_one_ne_φ_n}
\uses{lm:φ_n_minus_one_ne_φ_n_e,
lm:φ_n_minus_one_ne_φ_n_f}
自然数
$
e, f, n
$
が
$
n \ne 0
$
と
$
e | n
$
または
$
f | n
$
を満たす時、
$
\varphi_{e,f}(n-1) = \varphi_{e,f}(n)
$.
\end{lemma}

\begin{lemma}
\label{lm:nat_mul_dvd}
\lean{nat_mul_dvd}
\leanok
任意の自然数$a,b,c$に対して、
$
ab | c
$
ならば
$
b | c
$。
\end{lemma}

\begin{lemma}
\label{lm:min_φinv_dvd}
\lean{min_φinv_dvd}
\leanok
$n = \min \varphi_{e,f}^{-1}(i)$
であれば、
$e | n$または$f | n$が成り立つ。
\end{lemma}

\begin{proof}
$e \not | n$であれば、
$\lfloor n / e \rfloor = \lfloor (n-1) / e \rfloor$である。
同様に、
$f \not | n$であれば、
$\lfloor n / f \rfloor = \lfloor (n-1) / f \rfloor$である。
従って、
$e \not | n$かつ$f \not | n$であれば、
$
\lfloor (n-1) / e \rfloor + \lfloor (n-1) / f \rfloor
= \lfloor n / e \rfloor + \lfloor n / f \rfloor
= i
$
となり、
$n$が最小値であることに矛盾する。
\end{proof}

\begin{lemma}
\label{lm:dvd_min_φinv}
\lean{dvd_min_φinv}
\uses{df:φ}
$n \in \varphi_{e,f}^{-1}(i)$
が
$n | e$または$n | f$を満たせば、
$n$は
$\varphi_{e,f}^{-1}(i)$
の最小元である。
\end{lemma}

\begin{proof}
$n \in \varphi_{e,f}^{-1}(i)$
が
$n | e$を満たせば、
$
\lfloor (n-1) / e \rfloor < \lfloor n / e \rfloor
$
なので、
$
\lfloor (n-1) / f \rfloor \le \lfloor n / f \rfloor
$
と併せて、
$\lfloor (n-1) / e \rfloor + \lfloor (n-1) / f \rfloor < i$
となり、$n$は
$\varphi_{e,f}^{-1}(i)$
の最小元である。
同様に、
$n \in \varphi_{e,f}^{-1}(i)$
が
$n | f$を満たせば、
$n$は
$\varphi_{e,f}^{-1}(i)$
の最小元である。
\end{proof}

\begin{lemma}
$n \in \varphi_{e,f}^{-1}(i)$
が
$\varphi_{e,f}^{-1}(i)$
の最小元であるための
必要十分条件は、
$e | n$または$f | n$である。
\end{lemma}

\begin{lemma}
\label{lm:gcd_div_min_φinv}
\lean{gcd_div_min_φinv}
\uses{lm:min_φinv_dvd,lm:nat_mul_dvd}
\leanok
$n$が$\varphi_{e \ell, f \ell}^{-1}(i)$の最小元の時、
$\ell | n$である。
\end{lemma}

\begin{proof}
$e \ell| n$または$f \ell| n$であるが、
どちらの場合も$\ell | n$である。
\end{proof}

$\nmin(e,f,i) \in \varphi_{e,f}^{-1}(i)$である。
従って、
$\nmin(e,f,i) \not \equiv -1 \mod e$かつ
$\nmin(e,f,i) \not \equiv -1 \mod f$かつ
$\lfloor \nmin(e,f,i) / e \rfloor + \lfloor \nmin(e,f,i) / f \rfloor = i$が成り立つ。

\chapter{Main argument}

$e, f$を互いに素な2以上の自然数、
$i$を自然数、
$\ell$を自然数とする。

\begin{lemma}
\label{lm:φinv_i_empty_i_mod_e_add_f}
\lean{φinv_i_empty_i_mod_e_add_f}
$
\varphi_{e \ell, f \ell}^{-1}(i) = \emptyset
$
ならば
$
i \equiv e + f - 1 \mod e + f
$
\end{lemma}

\begin{proof}
$
\varphi_{e \ell, f \ell}^{-1}(i) = \emptyset
$
の時、
$
n = \min \varphi_{e \ell, f \ell}^{-1}(i+1)
$
は
$
e \ell | n
$
かつ
$
f \ell | n
$
を満たす。
この時、
$
n = e f \ell k
$
と置くと、
\begin{align}
\varphi_{e \ell, f \ell}(n)
&= \lfloor n / e \ell \rfloor + \lfloor n / f \ell \rfloor \\
&= f k + e k \\
&= (e + f) k
\end{align}
かつ
\begin{align}
\varphi_{e \ell, f \ell}(n-1)
&= (e + f) k - 2
\end{align}
より
\begin{align}
i = (e + f) k - 1
\end{align}
すなわち
\begin{align}
i \equiv e + f - 1 \mod e + f
\end{align}
である。
\end{proof}

\begin{lemma}
\label{lm:i_mod_e_add_f_φinv_i_empty}
\lean{i_mod_e_add_f_φinv_i_empty}
$
i \equiv e + f - 1 \mod e + f
$
ならば
$
\varphi_{e \ell, f \ell}^{-1}(i) = \emptyset
$
\end{lemma}

\begin{proof}
$
i \equiv e + f - 1 \mod e + f
$
の時、
$
k = (i + 1) / (e + f)
$,
$
n = e f \ell k
$
と置くと、
\begin{align}
\varphi_{e \ell, f \ell}(n)
&= \lfloor n / e \ell \rfloor + \lfloor n / f \ell \rfloor \\
&= f k + e k \\
&= (e + f) k \\
&= i + 1
\end{align}
かつ
\begin{align}
\varphi_{e \ell, f \ell}(n-1)
&= i - 1
\end{align}
なので、
$
\varphi_{e \ell, f \ell}^{-1}(i) = \emptyset
$
となる。
\end{proof}

\section{Case I}

$
i \not \equiv e + f - 1 \mod e + f
$
と仮定する。
この時
\Cref{lm:φinv_i_empty_i_mod_e_add_f}
から
$
\varphi_{e \ell, f \ell}^{-1}(i)
\ne
\emptyset
$
である。

\begin{definition}
\label{df:main_lemma.n_min_l}
\lean{main_lemma.n_min_l}
\leanok
$
\nmin(e \ell, f \ell, i)
\coloneqq
\min \varphi_{e \ell, f \ell}^{-1}(i)
$
\end{definition}

\begin{lemma}
\label{lm:main_lemma.min_l_eq_l_mul_min_1}
\lean{main_lemma.min_l_eq_l_mul_min_1}
\uses{df:main_lemma.n_min_l}
\leanok
$
\nmin(e \ell, f \ell, i) = \ell \nmin(e, f, i)
$
\end{lemma}

\begin{lemma}
\label{lm:main_lemma.case_a}
\lean{main_lemma.case_a}
\leanok
$i + 1 \not \equiv e + f - 1 \mod e + f$の時
$
\nmin (e \ell, f \ell, i+1) - 1
\in
\varphi_{e \ell, f \ell}^{-1}(i)
$
かつ
$
\nmin (e \ell, f \ell, i+1)
\not \in
\varphi_{e \ell, f \ell}^{-1}(i)
$
\end{lemma}

\begin{proof}
$
i + 1 \not \equiv e + f - 1 \mod e + f
$
から
$
\varphi_{e \ell, f \ell}^{-1}(i+1) \ne \emptyset
$
であり、
その最小元として
$
\nmin (e \ell, f \ell, i+1)
$
が定義される。
$
\nmin (e \ell, f \ell, i+1) - 1
\le
\nmin (e \ell, f \ell, i+1)
$
と
$
\varphi_{e \ell, f \ell}
$
の単調性から
$
\varphi_{e \ell, f \ell}(\nmin (e \ell, f \ell, i+1) - 1)
\le
\varphi_{e \ell, f \ell}(\nmin (e \ell, f \ell, i+1))
$
である。
$
\varphi_{e \ell, f \ell}(\nmin (e \ell, f \ell, i+1) - 1)
=
\varphi_{e \ell, f \ell}(\nmin (e \ell, f \ell, i+1))
$
は
$
\nmin (e \ell, f \ell, i+1)
$
の最小性に反するので、
$
\varphi_{e \ell, f \ell}(\nmin (e \ell, f \ell, i+1) - 1)
<
\varphi_{e \ell, f \ell}(\nmin (e \ell, f \ell, i+1))
$
である。
$
\varphi_{e \ell, f \ell}(\nmin (e \ell, f \ell, i)) = i
$
なので
$
\nmin (e \ell, f \ell, i)
\le
\nmin (e \ell, f \ell, i+1)
$
であり、
$
\varphi_{e \ell, f \ell}(\nmin (e \ell, f \ell, i))
=
i
\ne
i+1
=
\varphi_{e \ell, f \ell}(\nmin (e \ell, f \ell, i+1))
$
から
$
\nmin (e \ell, f \ell, i)
<
\nmin (e \ell, f \ell, i+1)
$
である。
従って
$
\nmin (e \ell, f \ell, i)
\le
\nmin (e \ell, f \ell, i+1) - 1
$
であり、
$\varphi_{e \ell, f \ell}$の単調性から
$
i
\le
\varphi_{e \ell, f \ell}( \nmin (e \ell, f \ell, i))
\le
\varphi_{e \ell, f \ell}( \nmin (e \ell, f \ell, i+1) - 1 )
$
となる。
従って
$
\varphi_{e \ell, f \ell}( \nmin (e \ell, f \ell, i+1) - 1 )
= i
$
である。
\end{proof}

\begin{lemma}
\label{lm:main_lemma.case_b}
\lean{main_lemma.case_b}
\uses{lm:i_mod_e_add_f_φinv_i_empty}
$
i + 1 \equiv e + f - 1 \mod e + f
$
の時
$
\nmin (e \ell, f \ell, i+2) - 1
\in
\varphi_{e \ell, f \ell}^{-1}(i)
$
かつ
$
\nmin (e \ell, f \ell, i+2)
\not \in
\varphi_{e \ell, f \ell}^{-1}(i)
$
\end{lemma}

\chapter{Bibliography}

\def\cprime{$'$} \def\cprime{$'$}
\providecommand{\bysame}{\leavevmode\hbox to3em{\hrulefill}\thinspace}
\providecommand{\MR}{\relax\ifhmode\unskip\space\fi MR }
% \MRhref is called by the amsart/book/proc definition of \MR.
\providecommand{\MRhref}[2]{%
  \href{http://www.ams.org/mathscinet-getitem?mr=#1}{#2}
}
\providecommand{\href}[2]{#2}
\newcommand{\arXiv}[1]{\href{https://arxiv.org/abs/#1}{arXiv:#1}}

\begin{thebibliography}{APZv1}
\bibitem[APZv1]{2404.17301v1}
Nikolas Adaloglou, Federica Pasquotto, and Aline Zanardini,
\emph{Symplectic cohomology of {$cA_n$} singularities},
\arXiv{2404.17301v1}.
\bibitem[APZv2]{2404.17301v2}
Nikolas Adaloglou, Federica Pasquotto, and Aline Zanardini,
\emph{Symplectic cohomology of {$cA_n$} singularities},
\arXiv{2404.17301v2}.
\bibitem[EL23]{MR4648096}
Jonathan~David Evans and Yanki Lekili,
\emph{Symplectic cohomology of compound {D}u {V}al singularities},
Ann. H. Lebesgue \textbf{6} (2023), 727--765. \arXiv{2104.11713}
\bibitem[LU21]{MR4371540}
Yank{\i} Lekili and Kazushi Ueda, \emph{Homological mirror symmetry for {M}ilnor fibers of simple singularities}, Algebr. Geom. \textbf{8} (2021), no.~5, 562--586. \arXiv{2004.07374}
\bibitem[LU22]{MR4442683}
Yank\i Lekili and Kazushi Ueda, \emph{Homological mirror symmetry for {M}ilnor fibers via moduli of {$A_\infty$}-structures}, J. Topol. \textbf{15} (2022), no.~3, 1058--1106. \arXiv{1806.04345}
\bibitem[LU]{2406.15915}
Yank{\i} Lekili, Kazushi Ueda,
\emph{Homological mirror symmetry for Rabinowitz Fukaya categories
of Milnor fibers of Brieskorn-Pham singularities},
\arXiv{2406.15915}.
\bibitem[W]{W}
Freek Wiedijk,
\emph{The ``de {Bruijn} factor''},
last modified on 2012-03-01,
available at \url{https://www.cs.ru.nl/~freek/factor/}.
\end{thebibliography}
