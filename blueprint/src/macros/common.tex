% In this file you should put all LaTeX macros to be used
% both by the pdf version and the web version.
% This should be most of your macros.

\usepackage{mathtools}
\usepackage{cleveref}

% % Bilingual
% \newcommand{\en}[1]{#1 \\}
% \newcommand{\ja}[1]{#1}

% % English only
% \newcommand{\en}[1]{#1}
% \newcommand{\ja}[1]{}

% \theoremstyle{plain}
% % \newtheorem{theorem}{Theorem}[section]
% \newtheorem{theorem}{Theorem}
% \newtheorem{corollary}[theorem]{Corollary}
% \newtheorem{lemma}[theorem]{Lemma}
% \newtheorem{definition-lemma}[theorem]{Definition-Lemma}
% \newtheorem{claim}[theorem]{Claim}
% \newtheorem{sublemma}[theorem]{Sublemma}
% \newtheorem{proposition}[theorem]{Proposition}
% \newtheorem{conjecture}[theorem]{Conjecture}
% \newtheorem{assumption}[theorem]{Assumption}
% \newtheorem{step}{Step}

% \theoremstyle{definition}
% \newtheorem{definition}[theorem]{Definition}
% \newtheorem{remark}[theorem]{Remark}
% \newtheorem{question}[theorem]{Question}
% \newtheorem{problem}[theorem]{Problem}
% \newtheorem{condition}[theorem]{Condition}
% \newtheorem{example}[theorem]{Example}

% ----- Japanese only -----

\newcommand{\en}[1]{}
\newcommand{\ja}[1]{#1}

% \theoremstyle{plain}
\theoremstyle{definition}
% \newtheorem{theorem}{Theorem}[section]
\newtheorem{theorem}{定理}
\crefname{theorem}{定理}{定理}
\newtheorem{corollary}[theorem]{系}
\crefname{corollary}{系}{系}
\newtheorem{lemma}[theorem]{補題}
\crefname{lemma}{補題}{補題}
\newtheorem{proposition}[theorem]{命題}
\crefname{proposition}{命題}{命題}
\newtheorem{conjecture}[theorem]{予想}
\crefname{conjecture}{予想}{予想}
\newtheorem{assumption}[theorem]{仮定}
\crefname{assumption}{仮定}{仮定}
\newtheorem{step}{ステップ}
\crefname{step}{ステップ}{ステップ}

\newtheorem{definition}[theorem]{定義}
\newtheorem{remark}[theorem]{注意}
\newtheorem{problem}[theorem]{問題}
\newtheorem{condition}[theorem]{条件}
\newtheorem{example}[theorem]{例}

% ----- Common -----

\newcommand{\bC}{\mathbb{C}}
\newcommand{\bN}{\mathbb{N}}
\newcommand{\bQ}{\mathbb{Q}}
\newcommand{\bZ}{\mathbb{Z}}
\newcommand{\rI}{\mathrm{I}}
\newcommand{\rII}{\mathrm{I\!I}}
\newcommand{\rIII}{\mathrm{I\!I\!I}}
\newcommand{\Uv}{\check{U}}
\DeclareMathOperator{\SH}{SH}
